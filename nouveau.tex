\documentclass[12pt]{beamer}
\usepackage[utf8]{inputenc}
\usepackage[T1]{fontenc}
\usepackage[french]{babel}
\usepackage{amsmath, amssymb, physics}
\usepackage{graphicx}
\usepackage{hyperref}
\usepackage{caption}
\usepackage{subcaption}
\usepackage{booktabs}
\usepackage{array}
\usepackage{tikz}
\usepackage{xcolor}
\usepackage{listings}
\usepackage{multicol}
\usepackage{adjustbox}
\usepackage{ragged2e}

% ============================================
% CONFIGURATION DU THÈME
% ============================================
\usetheme{Madrid}
\usecolortheme{seahorse}
\setbeamertemplate{navigation symbols}{}

% Couleurs personnalisées
\definecolor{unstimblue}{RGB}{0, 70, 140}
\definecolor{unstimorange}{RGB}{255, 102, 0}
\definecolor{unstimgreen}{RGB}{0, 128, 0}

\setbeamercolor{structure}{fg=unstimblue}
\setbeamercolor{title}{fg=white, bg=unstimblue}
\setbeamercolor{frametitle}{fg=white, bg=unstimblue!80}
\setbeamercolor{block title}{fg=white, bg=unstimorange}
\setbeamercolor{block body}{bg=unstimblue!5}
\setbeamercolor{alerted text}{fg=unstimorange}

% En-tête et pied de page
\setbeamertemplate{headline}{}
\setbeamertemplate{footline}{
    \hfill
    \scriptsize
    \color{gray}
    \insertframenumber/\inserttotalframenumber
    \hspace{0.3cm}
    \vspace{0.2cm}
}

% ============================================
% CONFIGURATION DES LISTINGS
% ============================================
\lstset{
    language=C,
    basicstyle=\tiny\ttfamily,
    keywordstyle=\color{unstimblue},
    commentstyle=\color{gray},
    stringstyle=\color{unstimorange},
    numbers=left,
    numberstyle=\tiny\color{gray},
    backgroundcolor=\color{gray!5},
    frame=single,
    rulecolor=\color{gray!30},
    tabsize=2,
    breaklines=true,
    captionpos=b,
    showstringspaces=false,
}

% ============================================
% COMMANDES PERSONNALISÉES
% ============================================
\newcommand{\mkl}{\textbf{Intel MKL}}
\newcommand{\dsygv}{\texttt{dsygv}}
\newcommand{\code}[1]{\texttt{\color{unstimblue}#1}}
\newcommand{\highlight}[1]{\textcolor{unstimorange}{\textbf{#1}}}

% ============================================
% PAGE DE TITRE
% ============================================
\title{Utilisation d'Intel MKL en Calcul Scientifique}
\subtitle{Application aux Vibrations de Membranes Non Uniformes}
\author{
    AHOTONHOUN Aimé Césaire \\
    BONOU Justus \\
    HANDJEMEDJI Ezéchiel
}
\institute{
    ENSGMM-ABOMEY \\
    Université Nationale des Sciences Technologies, Ingénierie et Mathématiques \\(UNSTIM)\\
    \textbf{Sous la supervision de : Dr. Carlos AGOSSOU}
}
\date{Année Académique 2025-2026}
\logo{\includegraphics[height=1.3cm]{pictures/ensgmm.png}}

\begin{document}

% ============================================
% PAGE DE TITRE
% ============================================
\begin{frame}
    \titlepage
    \begin{tikzpicture}[remember picture, overlay]
        \draw[unstimorange, line width=1.5pt] 
            ([xshift=0.05cm, yshift=-0.05cm]current page.north west) 
            rectangle 
            ([xshift=-0.05cm, yshift=0.05cm]current page.south east);
    \end{tikzpicture}
\end{frame}

% ============================================
% TABLE DES MATIÈRES
% ============================================
\begin{frame}{Plan de la Présentation}
    \tableofcontents
\end{frame}

% ============================================
% SECTION 1: INTRODUCTION
% ============================================
\section{Introduction}
\begin{frame}{Contexte du Projet}
    \begin{columns}[T]
        \begin{column}{0.7\textwidth}
            \begin{block}{Les vibrations de membranes}
                \footnotesize
                Phénomène physique omniprésent dans:
                \begin{itemize}
                    \item Instruments de musique
                    \item Capteurs mécaniques
                    \item Isolation vibratoire
                    \item Biologie cellulaire
                \end{itemize}
            \end{block}
            
            \begin{block}{Objectifs principaux}
                \footnotesize
                \begin{enumerate}
                    \item Comprendre la physique des vibrations
                    \item Simulation numérique avancée
                    \item Maîtriser \mkl{}
                    \item Analyser et visualiser
                \end{enumerate}
            \end{block}
        \end{column}
        
        \begin{column}{0.30\textwidth}
            \begin{center}
                \vspace{2cm}
                \begin{tikzpicture}
                \draw[fill=unstimblue!20] (0,0) rectangle (3,0.8);
                \node at (1.5,0.4) {\textbf{BLAS}};
                \draw[fill=unstimorange!20] (0,1) rectangle (3,1.8);
                \node at (1.5,1.4) {\textbf{LAPACK}};
                \draw[fill=unstimgreen!20] (0,2) rectangle (3,2.8);
                \node at (1.5,2.4) {\textbf{Parallélisme}};
            \end{tikzpicture}
            \end{center}
        \end{column}
    \end{columns}
\end{frame}

% ============================================
% SECTION 2: PROBLÈME PHYSIQUE
% ============================================
\section{Problème Physique}
\begin{frame}{Membrane Uniforme: Solution Analytique}
    \begin{columns}[T]
        \begin{column}{0.48\textwidth}
            \begin{block}{Équation des ondes}
                \small
                \[
                \frac{\partial^2 u}{\partial t^2} = c^2 \Delta u
                \text{ avec }c = \sqrt{T/\rho}\]
            \end{block}

            \begin{block}{Conditions aux limites}
                \small
                \[
                u(0, y, t) = u(Lx , y, t) = 0\]
                \[ u(x, 0, t) = u(x, Ly , t) = 0
                \]
            \end{block}
            
            \begin{block}{Solution par séparation}
                \small
                \[
                u(x,y,t) = X(x)Y(y)G(t)
                \]
            \end{block}
        \end{column}
        
        \begin{column}{0.48\textwidth}
            \begin{block}{Modes propres de vibrations}
                \small
                \[
                \phi_{n,m}(x,y) = \sin\left(\frac{n\pi x}{L_x}\right) \sin\left(\frac{m\pi y}{L_y}\right)
                \]
            \end{block}
            
            \begin{block}{Fréquences propres}
                \small
                \[
                \omega_{n,m} = c\sqrt{\left(\frac{n\pi}{L_x}\right)^2 + \left(\frac{m\pi}{L_y}\right)^2}
                \]
            \end{block}
        \end{column}
    \end{columns}
\end{frame}
\begin{frame}{Interprétation physique}
    \begin{exampleblock}{Exemples de modes}
                \tiny
                \begin{itemize}
                    \item Mode (1,1): Toute la membrane dans le même sens
                    \item Mode (1,2): Une moitié monte, l'autre descend
                    \item Mode (2,1) : Même chose mais selon l’autre direction
                    \item Mode (2,2) : Quatre régions alternent haut/bas
                \end{itemize}
            \end{exampleblock}
\end{frame}

\begin{frame}{Membrane Non Uniforme: Modèle Réaliste}
    \begin{block}{Équation générale}
        \small
        \[
        -\nabla \cdot \big(p(x,y) \nabla u(x,y)\big) + q(x,y)u(x,y) = \lambda w(x,y)u(x,y)
        \]
    \end{block}
    
    \begin{columns}[T]
        \begin{column}{0.4\textwidth}
            \begin{block}{Conditions aux limites}
            \[u(x, y) = 0 \text{ sur } \partial\Omega \]
            \end{block}
        \end{column}
        
        \begin{column}{0.5\textwidth}
            \begin{block}{Coefficients variables}
                \footnotesize
                \begin{align*}
                    p(x,y) &= 1 + 0.5\sin(2\pi x)\cos(2\pi y) \\
                    w(x,y) &= 1 + 0.3xy \\
                    q(x,y) &= 50e^{-50[(x-0.5)^2+(y-0.5)^2]}
                \end{align*}
            \end{block}
        \end{column}
    \end{columns}
\end{frame}

% ============================================
% SECTION 3: DISCRÉTISATION
% ============================================
\section{Discrétisation Numérique}
\begin{frame}{Méthode des Différences Finies}
    \begin{block}{Maillage uniforme}
                \small
                \[
                h = \frac{1}{N+1}, \quad x_i = ih, \quad y_j = jh
                \]
            \end{block}
            \textbf{Discrétisation du Laplacien}
                \footnotesize
                \[
\nabla \cdot (p \nabla u) \approx \frac{1}{h^2} \left[ p_{i+\frac{1}{2},j} (u_{i+1,j} - u_{i,j}) - p_{i-\frac{1}{2},j} (u_{i,j} - u_{i-1,j}) \right]\]
\[+ \frac{1}{h^2} \left[ p_{i,j+\frac{1}{2}} (u_{i,j+1} - u_{i,j}) - p_{i,j-\frac{1}{2}} (u_{i,j} - u_{i,j-1}) \right]
\]\\
\textbf{Discrétisation complète}
    \footnotesize
                \begin{multline*}
-\frac{1}{h^2}\left[ p_{i+\frac{1}{2},j} (u_{i+1,j} - u_{i,j}) - p_{i-\frac{1}{2},j} (u_{i,j} - u_{i-1,j}) \right] \\
-\frac{1}{h^2}\left[ p_{i,j+\frac{1}{2}} (u_{i,j+1} - u_{i,j}) - p_{i,j-\frac{1}{2}} (u_{i,j} - u_{i,j-1}) \right]
+ q_{i,j} u_{i,j} = \lambda w_{i,j} u_{i,j}
\end{multline*}
\end{frame}
\begin{frame}{Discrétisation}
    \begin{block}{Problème discrétisé}
                \small
                \[
                A\mathbf{u} = \lambda B\mathbf{u}
                \]
                \begin{itemize}
                    \item $A$: Matrice de rigidité (creuse)
                    \item $B$: Matrice de masse (diagonale)
                    \item $\lambda = \omega^2$: Valeurs propres
                    \item $u$: Vecteur des déplacements aux nœuds
                \end{itemize}
            \end{block}
\end{frame}

% ============================================
% SECTION 4: IMPLÉMENTATION MKL
% ============================================
\section{Implémentation avec Intel MKL}
\begin{frame}{Solveurs de Valeurs Propres dans MKL}
    \begin{table}
        \centering
        \scriptsize
        \begin{tabular}{|l|l|l|}
        \hline
        \textbf{Solveur} & \textbf{Type,Problème} & \textbf{Application} \\
        \hline
        \texttt{dsyev} & Dense, standard & Petites matrices \\
        \texttt{dsygv} & Dense, généralisé & Notre choix \\
        \texttt{FEAST} & Creux, sélectif & Grands systèmes \\
        \texttt{PARDISO+} & Creux, standard & Problèmes standards \\
        \hline
        \end{tabular}
        \caption{Solveurs disponibles dans Intel MKL}
    \end{table}
    
    \begin{block}{Pourquoi \dsygv{}?}
        \footnotesize
        \begin{columns}[T]
            \begin{column}{0.5\textwidth}
                \begin{itemize}
                    \item Matrices symétriques réelles
                    \item Problème généralisé
                    \item Stabilité numérique
                \end{itemize}
            \end{column}
            \begin{column}{0.5\textwidth}
                \begin{itemize}
                    \item Tous les modes calculés
                    \item Simple à implémenter
                    \item Bon pour taille modérée
                \end{itemize}
            \end{column}
        \end{columns}
    \end{block}
\end{frame}

\begin{frame}[fragile]{Code: Utilisation de \texttt{dsygv}}
\scriptsize
\begin{lstlisting}[basicstyle=\scriptsize\ttfamily]
#include <mkl.h>

int main() {
    int n = 2500;      // 50x50 points
    int itype = 1;     // A*x = lambda*B*x
    char jobz = 'V';   // Valeurs et vecteurs propres
    char uplo = 'U';   // Stockage triangle supérieur
    double *A = (double*)mkl_malloc(n*n*sizeof(double), 64);
    double *B = (double*)mkl_malloc(n*n*sizeof(double), 64);
    double *w = (double*)mkl_malloc(n*sizeof(double), 64);
    int lwork = -1;
    double work_query;
    dsygv(&itype, &jobz, &uplo, &n, A, &n, B, &n, 
          w, &work_query, &lwork, &info);
    lwork = (int)work_query;
    double *work = (double*)mkl_malloc(lwork*sizeof(double), 64);
    dsygv(&itype, &jobz, &uplo, &n, A, &n, B, &n, 
          w, work, &lwork, &info);
          
    mkl_free(A); mkl_free(B); mkl_free(w); mkl_free(work);
    return 0;
}
\end{lstlisting}
\end{frame}

% ============================================
% SECTION 5: RÉSULTATS
% ============================================
\section{Résultats et Visualisation}
\begin{frame}{Visualisation des Modes Propres}
    \begin{figure}
        \begin{subfigure}{0.33\textwidth}
            \centering
            \includegraphics[width=\textwidth]{mode_01.png}
            \caption*{Mode 1}
        \end{subfigure}
        \begin{subfigure}{0.33\textwidth}
            \centering
            \includegraphics[width=\textwidth]{mode_02.png}
            \caption*{Mode 2}
        \end{subfigure}
        \begin{subfigure}{0.33\textwidth}
            \centering
            \includegraphics[width=\textwidth]{mode_03.png}
            \caption*{Mode 3}
        \end{subfigure}
        \end{figure}
        \end{frame}
        \begin{frame}{}
        \begin{figure}
        \begin{subfigure}{0.4\textwidth}
            \centering
            \includegraphics[width=\textwidth]{mode_04.png}
            \caption*{Mode 4}
        \end{subfigure}
        \begin{subfigure}{0.4\textwidth}
            \centering
            \includegraphics[width=\textwidth]{mode_05.png}
            \caption*{Mode 5}
        \end{subfigure}
        \caption{Les cinq premiers modes de vibration de la membrane non uniforme}
    \end{figure}
\end{frame}

% ============================================
% SECTION 6: PERFORMANCES
% ============================================
\section{Performances et Optimisation}
\begin{frame}{Analyse des Performances}
    \begin{columns}[T]
        \begin{column}{0.45\textwidth}
            \begin{block}{Résultats pour N=50}
                \footnotesize
                \begin{itemize}
                    \item \highlight{Taille}: $2500 \times 2500$
                    \item \highlight{Mémoire}: $\sim 50$ Mo
                    \item \highlight{Temps}: 8.877 s
                    \item \highlight{Threads}: 4
                    \item \highlight{Précision}: Double
                \end{itemize}
            \end{block}
            \end{column}
            \begin{column}{0.45\textwidth}
            \begin{alertblock}{Limitations}
                \footnotesize
                \begin{itemize}
                    \item Stockage dense d'une matrice creuse
                    \item 99.8\% de zéros stockés
                    \item Non scalable pour N $>$ 100
                \end{itemize}
            \end{alertblock}
        \end{column}
\end{columns}
\begin{block}{Pour les grands systèmes}
                \footnotesize
                \begin{enumerate}
                    \item Format CSR : Stocker seulement les éléments non nuls
\item Solveur FEAST : Extraire sélectivement les valeurs propres désirées
\item Solveur itératif : Méthode de Lanczos ou Arnoldi pour les très grands
                \end{enumerate}
            \end{block}
\end{frame}
% ============================================
% CONCLUSION
% ============================================
\section{Conclusion}
\begin{frame}{Récapitulatif}
    \begin{columns}[T]
        \begin{column}{0.5\textwidth}
            \begin{block}{Accomplissements}
                \footnotesize
                \begin{itemize}
                    \item Modélisation physique complète
                    \item Discrétisation par différences finies
                    \item Implémentation avec \mkl{}
                    \item Résolution numérique efficace
                    \item Visualisation des résultats
                \end{itemize}
            \end{block}
            \vspace{0.5cm}
            \begin{center}
            \begin{tikzpicture}[scale=0.7]
                    \node[draw, fill=unstimblue!20, rounded corners, minimum width=2.5cm, minimum height=0.7cm, font=\tiny] 
                        at (0,0) {\textbf{Physique}};
                    \vspace{0.1cm}
                    \node[draw, fill=unstimorange!20, rounded corners, minimum width=2.5cm, minimum height=0.7cm, font=\tiny] 
                        at (0,-0.9) {\textbf{Mathématiques}};
                    \vspace{0.1cm}
                    \node[draw, fill=unstimgreen!20, rounded corners, minimum width=2.5cm, minimum height=0.7cm, font=\tiny] 
                        at (0,-1.8) {\textbf{Informatique}};
                \end{tikzpicture}    
            \end{center}
        \end{column}
        
        \begin{column}{0.4\textwidth}
            \begin{center}
            \begin{block}{Compétences développées}
                \footnotesize
                \begin{itemize}
                    \item Calcul scientifique haute performance
                    \item Utilisation de BLAS/LAPACK
                    \item Analyse numérique
                    \item Visualisation scientifique
                \end{itemize}
            \end{block}
            \end{center}
        \end{column}
    \end{columns}
\end{frame}

\begin{frame}{Dépôt GitHub et Références}
    \centering
    \LARGE
    \textbf{Code Source Complet}
    
    \vspace{0.5cm}
    
    \normalsize
    \href{https://github.com/AyomHead/projet_calcul_scientifique}
    {\color{unstimblue}\underline{\texttt{github.com/AyomHead/projet\_calcul\_scientifique}}}
    
    \vspace{0.8cm}
    
    \begin{columns}[T]
        \begin{column}{0.48\textwidth}
            \begin{block}{Contenu}
                \footnotesize
                \begin{itemize}
                    \item Code C avec MKL
                    \item Scripts Python
                    \item Données numériques
                    \item Visualisations
                    \item Document LaTeX
                \end{itemize}
            \end{block}
        \end{column}
        
        \begin{column}{0.5\textwidth}
            \begin{block}{Références}
                \footnotesize
                \begin{itemize}
                    \item Intel MKL Documentation
                    \item LAPACK User's Guide
                    \item Numerical Linear Algebra
                    \item Finite Element Procedures
                \end{itemize}
            \end{block}
        \end{column}
    \end{columns}
\end{frame}

% ============================================
% FIN DE PRÉSENTATION
% ============================================
\begin{frame}
    \centering
    \Huge
    \color{unstimblue}
    \textbf{MERCI POUR VOTRE ATTENTION}
    
    \vspace{0.8cm}
    
    \Large
    \textbf{Vos questions ou approches sont les bienvenues.}
    
    \vspace{1cm}
    
    \normalsize
    \color{black}
    \begin{tabular}{cc}
        \includegraphics[height=3cm]{pictures/ensgmm.png} &
        \includegraphics[height=3cm]{pictures/logo.png}
    \end{tabular}
\end{frame}

\end{document}